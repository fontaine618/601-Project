\documentclass[bj, preprint]{imsart}
\RequirePackage[OT1]{fontenc}
\RequirePackage[colorlinks,citecolor=blue,urlcolor=blue]{hyperref}
\usepackage{amsthm,amsmath,natbib,booktabs,cleveref}

% put your definitions there:
\endlocaldefs
\input{../utils/preamble}


\begin{document}

\begin{frontmatter}

\title{{\Large STATS 601 Project Proposal:} \\ 
\bf Caught Looking: Analyzing Variation in Umpire Strike Zones}
\runtitle{Caught Looking}

\begin{aug}
\author{
\fnms{Simon} 
\snm{Fontaine}
\thanksref{a,e1}
\ead[label=e1,mark]{simfont@umich.edu}
},
\author{
\fnms{Moritz} 
\snm{Korte-Stapff}
\thanksref{a,e2}
\ead[label=e2,mark]{kortest@umich.edu}
}
\and
\author{
\fnms{Brian} 
\snm{Manzo}
\thanksref{a,e3}
\ead[label=e3,mark]{bmanzo@umich.edu}
}
\address[a]{Ph.D. Student, University of Michigan, Department of Statistics. West Hall, 1085 South University, Ann Arbor, MI, U.S.A., 48109. \printead{e1,e2,e3}}
\runauthor{S. Fontaine, M. Korte-Stapff and B. Manzo}
\affiliation{University of Michigan, Department of Statistics}
\end{aug}



%\begin{abstract}
%The team members for this project are Simon Fontaine, Moritz Korte-Stapff, and Brian Manzo, all team members are first year Ph.D. students in the Department of Statistics.
%\end{abstract}

% \tableofcontents
\end{frontmatter}

%\newpage

% You each have your own command to annotate the text:
% \simon{Simon's comments}
% \moritz{Moritz's comments}
% \brian{Brian's comments}

% Proposal
% Each team submits to both the instructor and the GSI via email a one-page project proposal, including information about team members, the data set, and the methods you intend to apply with a brief justification.
% The deadline for the proposal is Monday, March 30


%------------------------------------------------------------------------------
\section{Introduction}\label{sec:intro}
As technology advances, Major League Baseball (MLB) has faced increased pressure from fans, coaches, and players to use video technologies to aid umpires in making calls on the field, especially for the notoriously subjective ball and strike calls.\footnote{\url{https://www.youtube.com/watch?v=Akx2kBavZ9Y}}
With this project, we will assess the ability of umpires to make ball and strike calls that match the rulebook and that are consistent across different game situations. 
To conduct our analysis, will access preprocessed PITCHf/x data for the entire 2018 MLB season from Kaggle \citep{schale2019mlb}. 
PITCHf/x is a system that uses cameras mounted in every MLB stadium to track the speed and location of every pitched baseball. 
PITCHf/x is paired with other data about the game and players involved that will enable us to assess the umpires' consistency.
\vspace{-1.5mm}

%------------------------------------------------------------------------------
\section{Research questions}\label{sec:research}
The first question we are looking to answer is whether umpires' realized strike zones match the requirements prescribed by the official rules of baseball. 
Once we have estimated strike zones for each umpire, we can use simulation to assess their ``error rate''---the percentage of pitches they would classify as a ball or strike that, according to the rules of baseball, should have the opposite classification.
As time and space permit, we also are looking to answer secondary questions such as whether umpires' strike zones change when there are 2 strikes or 3 balls in the count, and whether umpires have different strike zones for star pitchers and batters than they do for mediocre players.
The general theme of each of these questions is how good are umpires at (consistently) making the correct ball and strike calls?
The worse the umpires perform in different game situations, the louder will be the calls for technology-aided officiating.
\vspace{-1.5mm}

%------------------------------------------------------------------------------
\section{Methodology}\label{sec:method}
Estimating strike zones from pitch location data is fundamentally a nonlinear classification problem. 
Given pitches that have an $x$ and $y$ coordinate, we want to classify the pitches in a rectangle in the middle of the grid as strikes, and anything too far away from the middle (outside the rectangle) as balls.
Some methods that we think would succeed in drawing a realistic decision boundary are kernel logistic regression, additive logistic regression, support vector machines, tree-based methods, and neural networks. 
Upon selecting the best method modeling the umpires' strike zones, we will construct low-dimensional encodings (using, e.g., kernel PCA or a neural network approach) that will be used in the main analysis.
To study the effect of game- and player-level information as well as the effect of the particular umpire on the strike zone, we will consider a multivariate regression approach where the response will be taken as the estimated strike zone.
Finally, we can compare the list of ``best'' umpires according to our analysis with the umpires who were selected by MLB to officiate in the playoffs to see if our rankings match those of the league's executives.


%------------------------------------------------------------------------------
\bibliographystyle{imsart-nameyear}
\bibliography{../utils/references}{}





\end{document}
